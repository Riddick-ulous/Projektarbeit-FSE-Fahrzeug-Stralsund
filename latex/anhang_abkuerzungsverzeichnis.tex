%% ++++++++++++++++++++++++++++++++++++++++++++++++++++++++++++
%% Anhang: Abkürzungsverzeichnis
%% ++++++++++++++++++++++++++++++++++++++++++++++++++++++++++++
%
%  Gerüst:
%  * Version 0.11
%  * Dipl.-Ing. Karsten Renhak, karsten.renhak@tu-ilmenau.de
%  * Fachgebiet Kommunikationsnetze, TU Ilmenau
%
%  Für Hauptseminare, Studienarbeiten, Diplomarbeiten
%
%  Autor           : Max Mustermann
%  Letzte Änderung : 31.12.2011
%

% Hier keine weiteren Änderungen vornehmen
\cleardoublepage
\addchap{Abkürzungsverzeichnis}
\begin{acronym}[Bash]
 %nutzung mit \ac{NAME}, bei erstem Gebrauch wird das Acronym automatisch ausgeschrieben
	\acro{EFS}{\markup{E}lektronische \markup{F}ahrwerks\markup{s}ysteme GmbH} 
	\acro{MPP}{\markup{M}odel \markup{P}redictive \markup{P}lanning} 
	\acro{MPC}{\markup{M}odel \markup{P}redictive \markup{C}ontrol} 
	\acro{USK}{\markup{U}mfeld \markup{S}ensor \markup{K}oordinatensystem} 
	\acro{FRK}{\markup{Fr}enet \markup{K}oordinatensystem}
	
	%\acro{}{\markup{} \markup{} \markup{}}
	\acro{DAU}{\markup{D}ümmster \markup{A}nzunehmender \markup{U}ser}
	\acro{CAN}{\markup{C}ontroller \markup{A}rea \markup{N}etwork}
	\acro{RPM}{\markup{R}evolutions \markup{P}er \markup{M}inute}
	\acro{PMSM}{\markup{P}ermanent \markup{M}agnet \markup{S}ynchron \markup{M}aschine}
	\acro{WIG}{\markup{W}olfram \markup{I}nert \markup{G}aß}
	\acro{FSG}{\markup{F}ormula \markup{S}tudent \markup{G}ermany}
	\acro{CTMD}{\markup{C}ell\markup{T}emperature \markup{M}easurement \markup{D}evice}
	\acro{PCB}{\markup{P}rinted \markup{C}ircuit \markup{B}oard}
	\acro{SOC}{\markup{S}tate \markup{O}f \markup{C}harge}
	\acro{LED}{\markup{L}ight \markup{E}mitting \markup{D}iode}
	\acro{PTC}{\markup{P}ositive \markup{T}emperature \markup{C}oeficient}
	\acro{NTC}{\markup{N}egative \markup{T}emperature \markup{C}oeficient}
	\acro{SSR}{\markup{S}olid \markup{S}tate \markup{R}elais}
	\acro{ElKo}{\markup{El}ektrolyt \markup{Ko}ndensator}
	\acro{MLCC}{\markup{M}ulti\markup{l}ayer \markup{C}eramic \markup{C}apacitor}
	\acro{WE}{\markup{W}ürth \markup{E}lektronik}
	\acro{DF}{\markup{D}issipation \markup{SF}aktor}
	\acro{ESR}{\markup{E}quivalenter \markup{S}erieller \markup{W}iederstand}
	\acro{DCM}{\markup{D}iscontinious \markup{C}onduction \markup{M}ode}
	\acro{CCM}{\markup{C}ontinious \markup{C}onduction \markup{M}ode}
	\acro{IC}{\markup{I}ntegrated \markup{C}ircuit}
	\acro{POR}{\markup{P}ower \markup{O}n \markup{R}eset}
	\acro{PWM}{\markup{P}ulse \markup{W}eiten \markup{M}odulation} 
	\acro{TS}{\markup{T}ractive \markup{S}ystem }
	\acro{OPV}{\markup{OP}erations \markup{V}erstärker } 
	\acro{HV}{\markup{H}och \markup{V}olt } 
	\acro{SPI}{\markup{S}erial \markup{P}eripheral \markup{I}nterface } 
	\acro{GPIO}{\markup{G}eneral \markup{P}urpuose \markup{I}nput \markup{O}utput} 
	\acro{ADC}{\markup{A}nalog \markup{D}igital \markup{C}onverter} 
	\acro{idR}{\markup{i}n \markup{d}er \markup{R}egel} 
	\acro{IMD}{\markup{I}nsulation \markup{M}easurment \markup{D}evice} 
	\acro{AIR}{\markup{A}ccumulator \markup{I}solation \markup{R}elais} 
	\acro{AMS}{\markup{A}ccumulator \markup{M}anagment \markup{S}ystem} 
	\acro{HV}{\markup{H}igh \markup{V}oltage}
	\acro{TSMP}{\markup{T}ractive \markup{S}ystem \markup{M}easuring \markup{P}oint}
	\acro{BSPD}{\markup{B}rake \markup{S}ystem \markup{P}lausability \markup{D}evice}
	\acro{LVMS}{\markup{L}ow \markup{V}oltage \markup{M}ain \markup{S}witch}
	\acro{LVS}{\markup{L}ow \markup{V}oltage \markup{S}ystem}
	\acro{LV}{\markup{L}ow \markup{V}oltage}
	\acro{TSAL}{\markup{T}ractive \markup{S}ystem \markup{A}ctive \markup{L}ight}
	\acro{AIR}{\markup{A}ccumulator \markup{I}nsulation \markup{R}elais}
	\acro{LTS}{\markup{L}ap \markup{T}ime \markup{S}imulation}
	\acro{LiFePo4}{\markup{Li}thium \markup{Fe}rro\markup{Po}lymere}
	\acro{Li-ion}{\markup{Li}thium \markup{Ion}en}
	\acro{HVD}{\markup{H}igh \markup{V}oltage \markup{D}isconnect}
	\acro{SDC}{\markup{S}hut \markup{D}own \markup{C}ircuit}
	
\end{acronym}

