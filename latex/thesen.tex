%% ++++++++++++++++++++++++++++++++++++++++++++++++++++++++++++
%% Thesen zur Ausarbeitung. Für Diplomarbeiten
%% ++++++++++++++++++++++++++++++++++++++++++++++++++++++++++++
%
%  Gerüst:
%  * Version 0.11
%  * Dipl.-Ing. Karsten Renhak, karsten.renhak@tu-ilmenau.de
%  * Fachgebiet Kommunikationsnetze, TU Ilmenau
%
%  Für Hauptseminare, Studienarbeiten, Diplomarbeiten
%
%  Autor           : Max Mustermann
%  Letzte Änderung : 31.12.2011
%

\chapter*{Thesen zur \artderausarbeitung}
%\addcontentsline{toc}{chapter}{Thesen zur \artderausarbeitung}
%\ihead[]{Thesen zur \artderausarbeitung}

\begin{enumerate}
\item Mit \LaTeX\ gesetzte Dokumente sehen überall
      gleich aus. Sie werden ähnlich wie HTML in Klartext
      geschrieben und anschließend mit Hilfe eines Konverters in
      Postscript- oder PDF"=Dateien gewandelt.
\item \LaTeX\ gibt es für alle wichtigen Betriebssysteme.
\item Die Benutzung einer integrierten Entwicklungsumgebung,
      beispielsweise {\ttfamily Kile} oder {\ttfamily TeXnicCenter},
      wird empfohlen.
\item Dieses Dokument ist Formatvorlage und Einstiegshilfe
      zugleich. Einfach den Text durch die eigene Ausarbeitung
      ersetzen.
\end{enumerate}

% Etwas Platz schaffen:
\section*{}

Ilmenau, den 31.\,12.\,2011\hfill \namedesautors
