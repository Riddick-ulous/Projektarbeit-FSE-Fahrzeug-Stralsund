%  Gerüst:
%  * Version 1.0
%  * B.Eng. Eric Gorkow
%  * Student, TU Ilmenau
%  * Januar 2020
%
%  Für Hauptseminare, Studienarbeiten, Diplomarbeiten
%
%  Autor           : Max Mustermann
%  Letzte Änderung : Morgen
%
\documentclass
[   twoside=false,     % Einseitiger oder zweiseitiger Druck?
fontsize=12pt,     % Bezug: 12-Punkt Schriftgröße
DIV=15,            % Randaufteilung, siehe Dokumentation "KOMA"-Script
BCOR=0mm,         % Bindekorrektur: Innen 17mm Platz lassen. Copyshop-getestet.
%    headsepline,
headsepline,  % Unter Kopfzeile Trennlinie (aus: headnosepline)
footsepline,  % Über Fußzeile Trennlinie (aus: footnosepline)
open=right,        % Neue Kapitel im zweiseitigen Druck rechts beginnen lassen
paper=a4,          % Seitenformat A4
abstract=true,     % Abstract einbinden
listof=totoc,      % Div. Verzeichnisse ins Inhaltsverzeichnis aufnehmen
bibliography=totoc,% Literaturverzeichnis ins Inhaltsverzeichnis aufnehmen
titlepage,         % Titelseite aktivieren
headinclude=true,  % Seiten-Head in die Satzspiegelberechnung mit einbeziehen
footinclude=false, % Seiten-Foot nicht in die Satzspiegelberechnung mit einbeziehen
numbers=noenddot   % Gliederungsnummern ohne abschließenden Punkt darstellen
]   {scrreprt}         % Dokumentenstil: "Report" aus dem KOMA-Skript-Paket
\usepackage[table]{xcolor} % Anpassung tooltip
\usepackage[active]{srcltx}
%\usepackage[activate=normal]{pdfcprot} % Optischer Randausgleich -> pdflatex!
\usepackage{ifthen}
\usepackage[ngerman]{babel}   % Neue Deutsche Rechtschreibung
%\usepackage[latin1]{inputenc} % Zeichencodierung nach ISO-8859-1
\usepackage[utf8]{inputenc}   %	Zeichencodierung nach UTF-8 (Unicode)
\usepackage[T1]{fontenc}
\usepackage{lmodern} %Schriftart Latim modern
\usepackage[T1]{url}
\usepackage[final]{graphicx}
\usepackage[automark]{scrlayer-scrpage}
\usepackage{setspace}
%\usepackage[first,light]{draftcopy} % Für Probedruck
\usepackage{tabularx}
\usepackage{tikz}

\usepackage[plainpages=false,pdfpagelabels,hypertexnames=false]{hyperref}

%%% Modifications
\setlength{\parindent}{0em}%Einrücken der ersten Zeile eines neuen Absatz 0=keiner
\usepackage{float}
\usepackage[section]{placeins}
\usepackage{pdfpages}
\restylefloat{figure}
\usepackage[printonlyused,nohyperlinks]{acronym} % Anpassung
\usepackage{amsmath} % Anpassung
\usepackage{pdfcomment}% Anpassung tooltip
\usepackage{listings}
\lstset{numbers=left,stepnumber=2,, numberstyle=\tiny, numbersep=5pt, basicstyle=\small,}
% used Language
\lstset{language=Matlab}
\usepackage{hyperxmp} % metadata addon

%%% Matlab2tikz
\usepackage{pgfplots}
\pgfplotsset{compat=newest}
%% the following commands are needed for some matlab2tikz features
\usetikzlibrary{plotmarks}
\usetikzlibrary{arrows, arrows.meta, shapes.geometric}
\usepgfplotslibrary{patchplots}
\usepackage{grffile}
\usepackage{subfig}

\usepackage{attachfile} 

\usepackage{rotating}

\tikzset{
	font={\upshape\selectfont}}
\newcommand\tikzmark[2][]{
	\tikz[remember picture,baseline=(#1.base),align=left]{\node[minimum width=\hsize](#1){$#2$};}
}

% Tiefe der Kapitelnummerierung beeinflussen
\setcounter{secnumdepth}{3} % Tiefe der Nummerierung
\setcounter{tocdepth}{2}    % Tiefe des Inhaltsverzeichnisses

\usepackage[normalem]{ulem}
\newcommand{\markup}[1]{\textbf{#1}}

% Seitenlayout festlegen. Hier nichts ändern!
\pagestyle{scrplain}
\ihead[]{\headmark}
\ohead[]{\pagemark}
\chead[]{}
\ifoot[]{}
\ofoot[]{\scriptsize \artderausarbeitung\ \namedesautors}
\cfoot[]{}
\renewcommand{\titlepagestyle}{scrheadings}
\renewcommand{\partpagestyle}{scrheadings}
\renewcommand{\chapterpagestyle}{scrheadings}
\renewcommand{\indexpagestyle}{scrheadings}

% Abschnittsweise Nummerierung anstatt fortlaufend. Hier nichts ändern!
\makeatletter
\@addtoreset{equation}{chapter}
\@addtoreset{figure}{chapter}
\@addtoreset{table}{chapter}
\renewcommand\theequation{\thechapter.\@arabic\c@equation}
\renewcommand\thefigure{\thechapter.\@arabic\c@figure}
\renewcommand\thetable{\thechapter.\@arabic\c@table}
\makeatother

% Quelltextrahmen, klein. Hier nichts ändern!
\newsavebox{\inhaltkl}
\def\rahmenkl{\sbox{\inhaltkl}\bgroup\small\renewcommand{\baselinestretch}{1}\vbox\bgroup\hsize\textwidth}
\def\endrahmenkl{\par\vskip-\lastskip\egroup\egroup\fboxsep3mm%
	\framebox[\textwidth][l]{\usebox{\inhaltkl}}}

% Quelltextrahmen, normale Groesse. Hier nichts ändern!
\newsavebox{\inhalt}
\def\rahmen{\sbox{\inhalt}\bgroup\renewcommand{\baselinestretch}{1}\vbox\bgroup\hsize\textwidth}
\def\endrahmen{\par\vskip-\lastskip\egroup\egroup\fboxsep3mm%
	\framebox[\textwidth][l]{\usebox{\inhalt}}}

% Sonstige Befehlsdefinitionen hier ablegen.
\newcommand{\entspricht}{\stackrel{\wedge}{=}}

% Tabellenspaltendefinitionen mit fester Breite --> somit Zeilenumbruch innerhalb einer Zelle möglich
% aus http://www.torsten-schuetze.de/tex/tabsatz-2004.pdf
\usepackage{array, booktabs}
\newcolumntype{f}{>{$}l<{$}}
\newcolumntype{n}{>{\raggedright}l}
\newcolumntype{N}{>{\scriptsize}l}
\newcolumntype{v}[1]{>{\raggedright\hspace{0pt}}m{#1}}
\newcolumntype{V}[1]{>{\scriptsize\raggedright\hspace{0pt}}m{#1}}
\newcolumntype{Z}[1]{>{\raggedright\centering}m{#1}}
\newcolumntype{k}[1]{>{\raggedright}p{#1}}
% ergibt Tabllenspalte fester Breite, linksbündig
% Umbruch innerhalb der Zelle mit \\, neue Tabellezeile mit \tabularnewline
% \addlinespace für Gruppentrennung (aus \texttt{booktabs.sty})