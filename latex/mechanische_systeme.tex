% Autor: Lukas Deeken
% Letzte Bearbeitung: 01.05.2022

\chapter{Mechanische Systeme}

\section{Antriebslayout}
2WD ein motor (diff)
2Wd zwei motor (gewähltes konzept, Torque vektoring/hinterachslenkung)
4WD (Komplexität ungefederte massen etc.)

\section{Packaging}
Packaging leistungskennzahl. Volumenfüllungsgrad, schwerpunkt position, wartungsaufwand

\section{Systeme}

\subsection{Kühlung} (zusammen mit Julian Vogt und???)
Kühlungslayout (tiksedit grafik)


\subsubsection{Radiator \& Lüfter} (zusammen mit Julian Vogt und???)
Radiator berechnung
Lüfter auswahl (propeller umrechner etc)

Berechnung über Vorjahresfahrzeug Ähnlichkeitstheorie mit Absicherung über Vergleichsrechnungen



\subsubsection{Wasserpumpe und Schläuche} (zusammen mit Julian Vogt und???)
Wasserpumpen berechnung (kühlsystem berechnung, Druckabfälle etc.)



\begin{figure}
	\centering
	\includegraphics[width=0.7\linewidth]{bilder/Kühlsystemkennlinie}
	\caption{}
	\label{fig:kuhlsystemkennlinie}
\end{figure}


\subsection{Getriebe} (Michel und Linus)
zahnrad auslegung
Zahnrad fertigung
Kettentrieb alternative
Wellen auslegung
FEM detour

\subsubsection{Outbound vs Inbound}
Radnabenmotor vs interner motor
Antriebswellen
ungeferte massen
packaging im rad problem(Planetengetriebe) fertigungsaufwand

\subsubsection{Gussgehäuse vs Fräsgehäuse vs Schweißgehäuse} (Flo Irle)
SES anforderungen
Flexural rigidity
E modul vs yield strenght welche verbnesserungen bringen wo was


\subsubsection{Antriebswellen und Tripoden} Störle und schrang
excel tabellen und stuff von Störle und schrang

FEM sim bilder von Schrang
FEM konvergenz
netzunabhängigkeit
Kräfte richtig antragen
Feste flächen richtig wählen
kräfte richtig berechnen
Kontaktflächen bestimmen
Feinheitsgrade des netz
Inventor Casual fixe abschätzung oder einfache probleme vs Ansys Profi tool für komplexe verlässliche analyse
