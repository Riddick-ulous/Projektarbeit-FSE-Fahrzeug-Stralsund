% Autor: Lukas Deeken
% Letzte Bearbeitung: 01.05.2022

\chapter{Elektromechanische Systeme}

\section{Akkumulator}

\subsection{Zellenauswahl}

Wichtig bei der zellenauswahl ist das stets jede individuelle zelle für sich begutachtet werden muss. es gibt bei den diversen Bauformen und chemischen Zusammensetzungen gewissen Tendenzen welche im folgen erläutert werden. Jedoch ist die Überlappung dieser Eigenschaften in der Regel so groß das sich augenscheinlich vollkommen unterschiedliche Zellen für einen ähnlichen Einsatzzweck eignen.

\subsubsection{Vergleich der Speicherarten}

im nachfolgenden wird die zuerst die Energie berechnet die ein klassiches Formula Studentfahrzeug bei einem typischen bremsvorgang freisetzt und damit die enrgie die mann mindestens speichern können müsste um mit der speciherform auf sinnvolle art und weise eine rekuperation umszusetzten. Im anschluss wird diese energie in eine ungefähre masse an speicherelementen umgesetzt um zu zeigen inwiefern sich diese form der enrgiespeicherung für den einsatz eignet. im nachfolgenden wird die masse an speciherelementen bestimmt um 6 Kwh energie zu speichern da dies der typsiche energieverbrauch eines formula student fahrzueuges im Endurance ist. dieser wert wurde im rahmen eines benchmarkings mit den fahrzeuigen anderer teams über die letzten jahre 2016 bis 2019 errechnet.

Gewicht 220kg
Anfangsgeschwindigkeit 100 kmh
Endgeschwindigkeit 20 kmh
effizienz des antriebes 0.85


Physikalische Speicher (Kondensatoren)

	Kondensatoren erreichen ein sehr hohes leistungsgewicht, zeichnen sich jedoch durch eine geringe enrgiedichte aus, sowohl gravimetrisch als auch volumetrisch. daher eignet sich diese Form der enrgieseicherung nur um kurfristige transienten zu glätten aber nicht um gar ganze bremsvorgänge an energie zu speichern

Thermische Speicher (Salzakkumulator)

	sind im rahmen der formula student verboten Stand 2022, daher wird hier nicht weiter auf diese form des energiespeichers eingegangen

Mechanische Speicher (Schwungrad)

	Zeichnung sich durch relativ gute energiedichte als auch leistungdichte aus und bilden damit wahrscheinlich am ehesten eine realistische form des kurfristigen speichers für ein formula student fahrzeug. Jedoch sind solche systeme sehr komplex sowohl mechanisch elektrisch als auch regelungstechnisch im vergleich zu den anderen systemen. Die lagerung und sichere unterbringung des schwungrades in einem formel fahrzeug birgt große technische herausforderungen

Chemische Speicher (Klassische Akkuzelle)

	Der typische im Rahmen der formula studnet von allen teams eingesetzte energiespeicher. In der verfügbaren bandbreite findet man so ziemlich das optimal an leistungs als auch energiedichte.

\subsubsection{Runde vs Pouch vs Prismatische Zellen}
%tabelle
(
	Puch zelle

in der regelung höhere packungsdichte möglich damit höherte volumetrische enrgie und lkeistungsdichte
in der regel weniger zellen weniger als 300 manschmal sogar nur 150
weiches gehäuse ist leicht zu beschädigen, bedarf vorischtiger umgang 
aufblähen beim laden und entladen muss bei konstruktion berücksichtigt werden sonst platzenb der zellen möglich


Rundzelle

geringere fertigungstoleranzen durch serienfertigung idr kein matching erforderlich
hoher grad an standardisierung damit folgen mechanische austauschbarkeit und gute marktverfügbarkeit
Hartes gehäuse damit geringe wahrscheinlichkeit von penetrastion durch spitze objekte
bedarf in der regel sehr vieler zellen 600 und mehr, daher hohe mechanische komplexität

Prismatische Zellen

vorgefertigtes paket aus rund oder pouchzellen
sehr wenige zellen kleiner 150
sehr geringe mechanische komplexität da das paket in der regel mit elektrischen und mechanischen anbindungspunkten kommt meist sind auch schon temperatur sensoren integriert
meist jedoch sehr schwer aufgrund der ausrichtung auf industrielle bedürfnisse


Im rahmen des TY22 haben wir uns für den einsatz von rundzellen entschieden da diese nach unserem kenntnisstand gravimetrisch die höchste energiedichte liefern wir uns langfristig auf ein konzept festelgen wollten und so bei einsatz einer neuen akkuztelle nur gerinfügige änderungen an dem akku machen müssen sofern das 18650 format weiterhin populär bleibt. Außerdem war dies im rahmen der lieferschwierigkeiten im bereich der akuzellen im jahr 2021 die beste option um tatsächlich auch an akkuzellen für den bau des fahrzeuges zu kommen
)

\subsubsection{Zellchemie und Rekuperation}

Im folgenden eine tabellarische gegenüberstellung von \acfirst{LiFePo4} zellen und \acfirst{Li-ion} Zellen. Diese Tabelle basiert auf einer Sichtung von mehr als 30 verschiedenen Akkuzellen welche im rahmen des Projektes auf ihre Eignung für den Einsatz im Fahrzeug geprüft wurden. Liion umfasst dabei ein konglomerat aus diversen zellchemien welches eigentlich auch lifepo4 mit einschließt. Zur vereinfachung des vergleiches wurden alle liion chemieen mit einem typ. arbeitsbereich von 3-4,2 hierunter zusammengafasst. Die hierbei aufgrund der hohen löeistungsdichte am häufigsten vertretene Chemie ist LiNiMnCoO2



In der analyse ergibt sich das bild das sich \acfirst{LiFePo4} zellen für ein konzept mit hohem rekupoerationsanteil aber niederiger gesamtkapazität eignet während sich liion zellen für ein konzepot mit niedrigerem rekuperationsanteil und hoher gesamtkapazität eignen. Weiterhin muss hier berücksichtigt werden das Lifepo4 zellen meist ein niederigers temperaturmlimit beim laden als beim entladen haben was im betrieb zu einem vorzeitigen ausfall der rekuperation durch zu hohe akkutemperaturen führten kann. daher ist das temperatur managment hier von besoinderer bedeutung.

Das konzept mit hohen rekuströmen ist nur beim AWD Fahrzeug sinvoll anwendbar da hier auch die gesamte bremsenergie, abzüglich der verlsute im antriebsstran und einiger spitzenlasten welche die mechanische bremsanlage abfangen muss, verfügbar ist. Aufgrund der hohen komplexität des AWD systemes wurde beim TY22 auf ein 2WD System gesetzt. Daher ist der einsatz von konkret LiNiMnCoO2 zellen am ehesten sinvoll.

\subsubsection{Die \glqq Ideale\grqq Akkuzelle}

\section{Elektromotor}

\section{Wechselrichter}

\section{Kabelbaum}

\subsection{Kabeldimensionierung}

\subsection{Sicherungsauslegung}

\subsection{Steckverbinder Auswahl}

\subsection{HVD}

\subsection{AIR}

\section{Shutdowncircuit}

\section{Ladesystem / Handcart}

