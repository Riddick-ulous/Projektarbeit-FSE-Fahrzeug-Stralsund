% Autor: Lukas Deeken
% Letzte Bearbeitung: 01.05.2022

\chapter{Elektrische Systeme} (zusammen mit Leon Löser)
Hier Geht mein dank an die Alumni aus dem Bereich der Elektrotechnik die diese ganze Reise mit ihrer fachlichen Refferenz und beispiellosen motivation erst möglich gemacht habe. Hervcorzuheben ist auch das schier endlose mas an gedult in anbetracht der Steilen lernkurve innerhalkb des teams. Besonders hervorzuheben sind hierbei Leon Löser, Eric Gorkow und Axel Lange. Danke euch.
\acfirst{IMD}
\section{Akkumulator}
Akkumulaotr layout und connecxtions
	AMS DCDC IMD TS Sensors AIR etc. mit visio

\subsection{AMS Master und Slave}
Layout welche funktionen sind wo, wozu gibt es funktionen etc.
Sprich Precharge auf AMS Master

\subsubsection{Precharge}


\subsubsection{AIR Detection}


\subsubsection{AMS}
Konkret BMS selber sprich LTC6811 etc

\subsubsection{HV Indicator}
Funktionsprinzip erklären Viper chip

\subsubsection{HV Messung}

\subsubsection{IMD Monitoring}

\subsubsection{Strommessung}
Isabellenhütte sensor

\subsection{HV DCDC}
Visio blockmodell
Erklärung ACFC
Berechnung excel etwas aufhübschen und anhängen
Teilberechnungen erklären

\section{HV Distribution}

\subsection{TSMP}
Die Tractive System Measuring Points befinden sich seitlich am Fahrzeug wo auch der Hauptschalter zu finden ist. Sie stellen eine genormte Schnittstelle dar um mit einem Duspol, oder Multimeter an die Spannung des Tractive Systems gelangen zu können. Hierbei müssen laut Regelwerk geschirmte Bananenstecker eingesetzt werden. Weiter muss für die Buchsen am Fahrzeug eine Abdeckkappe oder Blindstecker vorgesehen werden. Zur Absicherung der TSMP müssen diese mit Widerständen in reihe an den HV Kreis angebunden werden. Das Regelwerk sieht hierbei in unserem Spannungsbereich 15k$\Omega$ vor. Der Kritische wert wonach die TSMP ausgelegt werden müssen ist das Leistungsrating, da diese auf kontinuierlichen Kurzschluss ausgelegt sein müssen.

Eine Formel zu Berechnung der Leistung am Widerstand ist folgende

\begin{equation}
	\label{eqn:Leistung am Wiederstand}
	\glsc{symb:P_elektrisch} = \glsc{symb:I}^{2} * \glsc{symb:R}
\end{equation}

Der Strom errechnet sich aus.

\begin{equation}
	\label{eqn:URI}
	\glsc{symb:U} = \glsc{symb:R}^{2} * \glsc{symb:I}
\end{equation}

Umgestellt nach I.

\begin{equation}
	\label{eqn:IUR}
	\glsc{symb:I} = \dfrac{\glsc{symb:U}} {\glsc{symb:R}}
\end{equation}

Da im Kurzschlussfall die Spannung über beide Widerstände anliegt, verdoppelt sich der Widerstand.

\begin{table}[h]
	\centering
	\begin{tabular}{|c|c|c|}
		\hline
		\multicolumn{3}{|c|}{Eingangsparameter} \\
		\hline
		\glsc{symb:R} & 15 & \ensuremath{k\Omega} \\
		\hline
		\glsc{symb:U} & 600 & \ensuremath{V} \\
		\hline
		\multicolumn{3}{|c|}{Ergebnisse} \\
		\hline
		\glsc{symb:I} & 20 & \ensuremath{mA} \\
		\hline
		\glsc{symb:P_elektrisch} & 12 & \ensuremath{W} \\
		\hline
	\end{tabular}
\end{table}

Daraus schlussfolgert sich das die 15k$\Omega$ Widerstände mit einem Leistungsrating von mindestens 12 W benötigt werden.

\subsection{BSPD}

VAC Sensorboard
Logik
\begin{figure}
	\centering
	\includegraphics[width=0.7\linewidth]{"bilder/BSPD Blockdiagramm"}
	\caption{}
	\label{fig:bspd-blockdiagramm}
\end{figure}

Die Aufgabe des BSPD ist es das Fahrzeug in dem Fall einer Störung des Gaspedales in einen Sicheren zustand zu überführen. Hierfür wird der Strom zu den Umrichtern als auch der Bremsdruck gemessen und beim eintreten eines im Regelwerk definierten schwellwertes für das gleichzeitige auftreten beider Signale muss das abschalten des Antriebes erfolgen. Das Folgende Blockdiagramm soll einen Überblick über den signalfluss ermöglichen.

Bei den Eingangsignalen handelt es sich um analoge spannungen. Für Sigfnalaufbereitung oder auch Digitalisierung der Signale werden Schitds Trigger einbgesetzt, Die Logik besteht aus diversen Logikgattern und die Set/reset Schaltung besteht aus RC Gliedern mit nachgeschaltetetn schmidt triggern. Beim Shutdowncircuit handelt es sich um ein Solid State Relay welches von der BSPD Logik schlussendlich angestuert werden soll, ein öffnen des Shutdwon circuit führt damnit zu einem Herunterfahren des Antriebes.

Zur Auslegung von Schmidt Triggern
\begin{figure}
	\centering
	\includegraphics[width=0.5\linewidth]{bilder/Schmitt-trigger-diagramm.png}
	\caption{}
	\label{fig:schmitt-trigger-diagramm}
\end{figure}

Die Funktionsweise eines Schmitt Trigger kann anhand des Bildes erkannt werden. Er ermöglicht es ein analoges Signal in ein Digitales umzuwandeln, dabei ist es möglich festzulegen welche Spannungspegel am Ausgang des Schmitt Trigger anliegen (\glsc{symb:U_HA} und \glsc{symb:U_LA}) und bei welchen Spannungspegeln der Trigger High respektive Low schalten soll (\glsc{symb:U_H}und \glsc{symb:U_L}). Das vorliegen unterschiedilcher Pegel zum Schalten wird Hysterese genannt. Grund für das vorliegen ist das verhindern des rapiden Umschaltens zwischen High und Low direkt an dem Schwellwert aufgrund von Signalrauschen.

\begin{figure}
	\centering
	\includegraphics[width=0.7\linewidth]{"bilder/TPS Failure detection"}
	\caption{}
	\label{fig:tps-failure-detection}
\end{figure}

\begin{figure}
	\centering
	\includegraphics[width=0.5\linewidth]{"bilder/nichtinvertierender Trigger"}
	\caption{}
	\label{fig:nichtinvertierender-trigger}
\end{figure}

Folgend Beispielhaft die Auslegung eines Nicht Invertierenden Schmitt Triggers wie er im Bild unten zu sehen ist. Die andere Ausführung ist die eines Invertierenden.

Zur Berechnung sollten sich \glsc{symb:U_HA} und \glsc{symb:U_LA} sowie \glsc{symb:U_H} und \glsc{symb:U_L} aus dem Betriebsfall ergeben. R\textsubscript{1} sowie R\textsubscript{3} können frei gewählt werden. R\textsubscript{1} ist hierbei der Verbund aus R\textsubscript{3} und R\textsubscript{4}.

Die beiden folgenden Gleichung liegen zu Grunde

\begin{equation}
	\label{eqn:Obere Hysteresespannung Schmitt Trigger}
	\glsc{symb:U_H} = \glsc{symb:U_ref} + (\glsc{symb:U_HA} - \glsc{symb:U_ref}) * \dfrac{R\textsubscript{1}} {R\textsubscript{1}+R\textsubscript{2}}
	mit R\textsubscript{1}=\dfrac{R\textsubscript{3}*R\textsubscript{4}}{R\textsubscript{3}+R\textsubscript{4}}
\end{equation}

Und

\begin{equation}
	\label{eqn:Untere Hysteresespannung Schmitt Trigger}
	\glsc{symb:U_L}=\glsc{symb:U_ref}-(\glsc{symb:U_ref}-\glsc{symb:U_HA})*\dfrac{R\textsubscript{1}}{R\textsubscript{1}+R\textsubscript{2}}
\end{equation}

Mit den folgenden Gleichungen lassen sich R\textsubscript{2} und R\textsubscript{4} bestimmen

\begin{equation}
	\label{eqn:Berechnung R2 Schmitt Trigger}
	R\textsubscript{2} = R\textsubscript{1} * \dfrac{\glsc{symb:U_HA} - \glsc{symb:U_LA}} {\glsc{symb:U_H} - \glsc{symb:U_L}}
\end{equation}

\begin{equation}
	\label{eqn:Berechnung Uref Schmitt Trigger}
	\glsc{symb:U_ref} = (\glsc{symb:U_H} - \glsc{symb:U_LA}) * \dfrac{R\textsubscript{2}} {R\textsubscript{1} + R\textsubscript{2}} + \glsc{symb:U_LA}
\end{equation}

\begin{equation}
	\label{eqn:Berechnung R4 Schmitt Trigger}
	R\textsubscript{4} = R\textsubscript{3} * \dfrac{\glsc{symb:VCC}-\glsc{symb:U_ref}} {\glsc{symb:U_ref}}
\end{equation}

\begin{figure}
	\centering
	\includegraphics[width=0.7\linewidth]{"bilder/BSPD Integrator"}
	\caption{}
	\label{fig:bspd-integrator}
\end{figure}

Zur Auslegung der Zeitsteuerung

Die Zeitsteuerung besteht aus einem Kondensator C2 welcher über den Widerstand R20 geladen wird, einer Diode D3 um Rückkopplung zu vermeiden, einem Widerstand R23 um den Kondensator langsam zu entladen, einem Spannungsfolger OP2 um die Schaltung von dem nachgeschalteten Schmitt Trigger zu entkoppeln und einem Transistor T3 der den Kondensator in kurzer Zeit bei Bedarf entladen kann.

Für die Berechnung ist die Ladezeit des Kondensators über den Widerstand R20 bis zur Schaltspannung des Schmitt Triggers zu ermitteln. Dies lässt sich mit folgender Gleichung Lösen. 

\begin{equation}
	\label{eqn:Ladezeit Kondensator}
	\glsc{symb:U_a}=\glsc{symb:U_e}*(1-\glsc{symb:e}^{\dfrac{1}{\glsc{symb:R}*\glsc{symb:C}}*\glsc{symb:t}})
\end{equation}

Alternativ kann der Schmitt Trigger aber auch auf 0,63 fache der Eingangsspannung gesetzt werden was der einfachen Zeitkonstante des RC Gliedes entspricht. Damit bestimmt sich C wie folgt.

\begin{equation}
	\label{eqn:Zeitkonstante Kondensator}
	\glsc{symb:C}=\dfrac{\glsc{symb:tau}}{\glsc{symb:R}}
\end{equation}

Mit Beispielhaften Auslegeparametern ergeben sich folgende Werte.

\begin{table}[h]
	\centering
	\begin{tabular}{|c|c|c|}
		\hline
		\multicolumn{3}{|c|}{Eingangsparameter} \\
		\hline
		\glsc{symb:tau} & 0,5 & \ensuremath{s} \\
		\hline
		\glsc{symb:R} & 100 & \ensuremath{k\Omega} \\
		\hline
		\multicolumn{3}{|c|}{Ergebnisse} \\
		\hline
		\glsc{symb:C} & 5 & \ensuremath{uF} \\
		\hline
	\end{tabular}
\end{table}

Das BSPD-Sensorboard hat zum Zweck einen Stromsensor zu kreieren der ein Stromsignal von 0-100A in ein Spannungssignal von 0.5-4.5V über den Bereich von 0-10A zu erzeugen. Die 0.5V Offset haben zum ziel eine Kurzschlusserkennung auf Masse als auch auf Versorgung zu ermöglichen. Der relevante Messbereich von 0-10A entsteht aus den Regelwerksanforderungen die vorgeben das der Fehlerzustand bei einer abgerufenen Leistung größer 5kW eingestellt werden muss was bei 600V in einem Strom von 8,3A resultiert. Dieser Bereich muss möglichst robust ausgewertet werden können.

Zur genauen Funktion, U1 ist ein Hall Effekt Stromsensor mit einem Übersetzungsfaktor von 1000. Heißt 10A ergeben 10mA Stromfluss am Ausgang. Der Widerstand R2 ist so gewählt das bei einem Strom von 10mA, 5V über den Widerstand abfallen und wir so in den Messbereich kommen. Der Widerstand R1 ermöglicht nun das konstante einspeisen von ca. 0,5V und die Diode das Begrenzen der max. Spannung auf 4,5V.

\begin{figure}
	\centering
	\includegraphics[width=0.7\linewidth]{"bilder/Sensorboard Schaltung"}
	\caption{}
	\label{fig:sensorboard-schaltung}
\end{figure}

\subsection{Discharge}
Die Discharge Schaltung soll bei Abschalten des Fahrzeug die Zwischenkreiskondensatoren entladen. Ziel ist es das System möglichst schnell in einen Spannungsfreien und damit sicheren Zustand zu überführen.
\begin{figure}
	\centering
	\includegraphics[width=0.7\linewidth]{bilder/Discharge}
	\caption{}
	\label{fig:discharge}
\end{figure}

Dies kann mit Hilfe von PTC Widerständen geschehen. Das Regelwerk sieht vor das der Zwischenkreis in maximal 5s auf 60VDC oder weniger zu bringen ist. Dies muss für 3 aufeinanderfolgende Entladevorgänge möglich sein. 

Die Ansteuerung erfolgt über den SDC welcher über den Steckverbinder C2 eingebunden ist. Von dort wird der Optokoppler U1 bestromt. Dieser Steuert Strom vom Gate des Mosfets Q1 weg Richtung TS- so das der Mosfet öffnet. Wenn der SDC geöffnet wird, steigt die Spannung am Gate auf 20V und der Mosfet steuert TS+ auf TS- über die PTC Wiederstände durch. Dadurch wird die Spannung im zwischenkreis in den PTC elementen abgebaut.

Die Formel für die Berechnung der PTC Elemenete ist dem Datenblatt für die PTCEL Serie der Firma Vishay zu entnehmen.

\begin{equation}
	\label{eqn:PTC Berechnung}
	\glsc{symb:N_PTC}=\dfrac{\glsc{symb:N_dump} * \glsc{symb:K} * \glsc{symb:C} * \glsc{symb:U}^{2}} {2 * \glsc{symb:R} * \glsc{symb:C_th} * (\glsc{symb:T_SW} - \glsc{symb:T_u})}
\end{equation}

\begin{table}[h]
	\centering
	\begin{tabular}{|c|c|c|c|}
		\hline
		\multicolumn{4}{|c|}{Eingangsparameter} \\
		\hline
		\glsc{symb:T_SW} & 130 & \ensuremath{°C} & Datenblatt Wiederstandsverlauf \\
		\hline
		\glsc{symb:K} & 1 & \ensuremath{-} & Datenblatt -> DC \\
		\hline
		\glsc{symb:C_th} & 2,3 & \ensuremath{J/K} & Datenblatt -> PTCEL17 \\
		\hline
		\glsc{symb:T_u} & 60 & \ensuremath{°C} & Worst Case \\
		\hline
		\glsc{symb:C} & 400 & \ensuremath{uF} & 2x DTI 500 Kapazität \\
		\hline
		\glsc{symb:U} & 600 & \ensuremath{V} & \\
		\hline
		\glsc{symb:N_dump} & 4 & \ensuremath{-} & min 3 Regelwerk\\
		\hline
		\multicolumn{4}{|c|}{Ergebnisse} \\
		\hline
		\glsc{symb:N_PTC} & 1,79 & \ensuremath{-} &  \\
		\hline
	\end{tabular}
\end{table}

Damit ergibt sich das 2 PTC`s des Typ 17R251 oder 17R501 verwendet werden müssen. 

Die Entladezeit kann wie beim BSPD über die bestimmt werden, in diesem Fall näherungsweise über die 3 fache Zeitkonstante. Damit ergibt sich eine Entladezeit von max. 1,2s.

\section{TSAL}

\subsection{Logik auf Discharge}

\subsection{Logik auf AMS Master}