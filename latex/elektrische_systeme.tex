% Autor: Lukas Deeken
% Letzte Bearbeitung: 01.05.2022

\chapter{Elektrische Systeme} (zusammen mit Leon Löser)
Hier Geht mein dank an die Alumni aus dem Bereich der Elektrotechnik die diese ganze Reise mit ihrer fachlichen Refferenz und beispiellosen motivation erst möglich gemacht habe. Hervcorzuheben ist auch das schier endlose mas an gedult in anbetracht der Steilen lernkurve innerhalkb des teams. Besonders hervorzuheben sind hierbei Leon Löser, Eric Gorkow und Axel Lange. Danke euch.
\acfirst{IMD}
\section{Akkumulator}
Akkumulaotr layout und connecxtions
	AMS DCDC IMD TS Sensors AIR etc. mit visio

\subsection{AMS Master und Slave}
Layout welche funktionen sind wo, wozu gibt es funktionen etc.
Sprich Precharge auf AMS Master

\subsubsection{Precharge}


\subsubsection{AIR Detection}


\subsubsection{AMS}
Konkret BMS selber sprich LTC6811 etc

\subsubsection{HV Indicator}
Funktionsprinzip erklären Viper chip

\subsubsection{HV Messung}

\subsubsection{IMD Monitoring}

\subsubsection{Strommessung}
Isabellenhütte sensor

\subsection{HV DCDC}
Visio blockmodell
Erklärung ACFC
Berechnung excel etwas aufhübschen und anhängen
Teilberechnungen erklären

\section{HV Distribution}

\subsection{TSMP}
TSMP Wiederstände auslegung und steckverbidner plus kappen.

\subsection{BSPD}

VAC Sensorboard
Logik
\begin{figure}
	\centering
	\includegraphics[width=0.7\linewidth]{"bilder/BSPD Blockdiagramm"}
	\caption{}
	\label{fig:bspd-blockdiagramm}
\end{figure}

Die Aufgabe des BSPD ist es das Fahrzeug in dem Fall einer Störung des Gaspedales in einen Sicheren zustand zu überführen. Hierfür wird der Strom zu den Umrichtern als auch der Bremsdruck gemessen und beim eintreten eines im Regelwerk definierten schwellwertes für das gleichzeitige auftreten beider Signale muss das abschalten des Antriebes erfolgen. Das Folgende Blockdiagramm soll einen Überblick über den signalfluss ermöglichen.

Bei den Eingangsignalen handelt es sich um analoge spannungen. Für Sigfnalaufbereitung oder auch Digitalisierung der Signale werden Schitds Trigger einbgesetzt, Die Logik besteht aus diversen Logikgattern und die Set/reset Schaltung besteht aus RC Gliedern mit nachgeschaltetetn schmidt triggern. Beim Shutdowncircuit handelt es sich um ein Solid State Relay welches von der BSPD Logik schlussendlich angestuert werden soll, ein öffnen des Shutdwon circuit führt damnit zu einem Herunterfahren des Antriebes.

Zur Auslegung von Schmidt Triggern
\begin{figure}
	\centering
	\includegraphics[width=0.5\linewidth]{bilder/Schmitt-trigger-diagramm.png}
	\caption{}
	\label{fig:schmitt-trigger-diagramm}
\end{figure}

Die Funktionsweise eines Schmitt Trigger kann anhand des Bildes erkannt werden. Er ermöglicht es ein analoges Signal in ein Digitales umzuwandeln, dabei ist es möglich festzulegen welche Spannungspegel am Ausgang des Schmitt Trigger anliegen (\glsc{symb:U_HA} und \glsc{symb:U_LA}) und bei welchen Spannungspegeln der Trigger High respektive Low schalten soll (\glsc{symb:U_H}und \glsc{symb:U_L}). Das vorliegen unterschiedilcher Pegel zum Schalten wird Hysterese genannt. Grund für das vorliegen ist das verhindern des rapiden Umschaltens zwischen High und Low direkt an dem Schwellwert aufgrund von Signalrauschen.

\begin{figure}
	\centering
	\includegraphics[width=0.7\linewidth]{"bilder/TPS Failure detection"}
	\caption{}
	\label{fig:tps-failure-detection}
\end{figure}

\begin{figure}
	\centering
	\includegraphics[width=0.5\linewidth]{"bilder/nichtinvertierender Trigger"}
	\caption{}
	\label{fig:nichtinvertierender-trigger}
\end{figure}

Folgend Beispielhaft die Auslegung eines Nicht Invertierenden Schmitt Triggers wie er im Bild unten zu sehen ist. Die andere Ausführung ist die eines Invertierenden.

Zur Berechnung sollten sich \glsc{symb:U_HA} und \glsc{symb:U_LA} sowie \glsc{symb:U_H} und \glsc{symb:U_L} aus dem Betriebsfall ergeben. R\textsubscript{1} sowie R\textsubscript{3} können frei gewählt werden. R\textsubscript{1} ist hierbei der Verbund aus R\textsubscript{3} und R\textsubscript{4}.

Die beiden folgenden Gleichung liegen zu Grunde

\begin{equation}
	\label{eqn:Obere Hysteresespannung Schmitt Trigger}
	\glsc{symb:U_H} = \glsc{symb:U_ref} + (\glsc{symb:U_HA} - \glsc{symb:U_ref}) * \dfrac{R\textsubscript{1}} {R\textsubscript{1}+R\textsubscript{2}}
	mit R\textsubscript{1}=\dfrac{R\textsubscript{3}*R\textsubscript{4}}{R\textsubscript{3}+R\textsubscript{4}}
\end{equation}

Und

\begin{equation}
	\label{eqn:Untere Hysteresespannung Schmitt Trigger}
	\glsc{symb:U_L}=\glsc{symb:U_ref}-(\glsc{symb:U_ref}-\glsc{symb:U_HA})*\dfrac{R\textsubscript{1}}{R\textsubscript{1}+R\textsubscript{2}}
\end{equation}

Mit den folgenden Gleichungen lassen sich R\textsubscript{2} und R\textsubscript{4} bestimmen

\begin{equation}
	\label{eqn:Berechnung R2 Schmitt Trigger}
	R\textsubscript{2} = R\textsubscript{1} * \dfrac{\glsc{symb:U_HA} - \glsc{symb:U_LA}} {\glsc{symb:U_H} - \glsc{symb:U_L}}
\end{equation}

\begin{equation}
	\label{eqn:Berechnung Uref Schmitt Trigger}
	\glsc{symb:U_ref} = (\glsc{symb:U_H} - \glsc{symb:U_LA}) * \dfrac{R\textsubscript{2}} {R\textsubscript{1} + R\textsubscript{2}} + \glsc{symb:U_LA}
\end{equation}

\begin{equation}
	\label{eqn:Berechnung R4 Schmitt Trigger}
	R\textsubscript{4} = R\textsubscript{3} * \dfrac{\glsc{symb:VCC}-\glsc{symb:U_ref}} {\glsc{symb:U_ref}}
\end{equation}



\begin{figure}
	\centering
	\includegraphics[width=0.7\linewidth]{"bilder/BSPD Integrator"}
	\caption{}
	\label{fig:bspd-integrator}
\end{figure}

Zur Auslegung der Zeitsteuerung

Die Zeitsteuerung besteht aus einem Kondensator C2 einer Diode D3 um Rückkopplung zu vermeiden, einem Spannungsfolger OP2 um die Schaltung von dem nachgeschalteten Schmitt Trigger zu entkoppeln und einem Transistor T3 der den Kondensator in kurzer Zeit bei Bedarf entladen kann. Die Auslegung erfolgt im Rahmen einer durch den Schmitt Trigger gewählte Schaltspannung. 

Für die Berechnung ist die Ladezeit des Kondensators über den Wiederstand R20 bis zur Schalktspannung des Schmitt Triggers zu ermittlen.

\begin{equation}
	\label{eqn:Ladezeit Kondensator}
	\glsc{symb:U_a}=\glsc{symb:U_e}*(1-\glsc{symb:e}^{\dfrac{1}{\glsc{symb:R}*\glsc{symb:C}}*\glsc{symb:t}})
\end{equation}

Das Sensorboard hat zum Zweck einen Stromsensor zu kreieren der ein Stromsignal von 0-100A in ein Spannungsignal von 0.5-4.5V über den Bereich von 0-10A zu erzeugen. Die 0.5V Offset haben zum ziel eine Kurzschlusserkennung auf Masse als auch auf Versorgung zu ermöglichen. Der relevante Messbereich von 0-10A entsteht aus den Regelwerksanforderungen die vorgeben das der Fehlerzustand bei einer abgerufenen Leistung größer 5kW eingestellt werden muss was bei 600V in einem Strom von 8,3A resultiert. Dieser Bereich muss möglichst robust ausgewertet werden können.

Zur genauen Funktion, U1 ist ein Hall Effekt Stromsensor mit einem Übersetzungsfaktor von 1000. Heißt 10A ergeben 10mA Stromfluss am Ausgang. Der Widerstand R2 ist so gewählt das bei einem Strom von 10mA, 5V über den Widerstand abfallen und wir so in den Messbereich kommen. Der Widerstand R1 ermöglicht nun das konstante einspeisen von ca. 0,5V und die Diode das Begrenzen der max. Spannung auf 4,5V.

\begin{figure}
	\centering
	\includegraphics[width=0.7\linewidth]{"bilder/Sensorboard Schaltung"}
	\caption{}
	\label{fig:sensorboard-schaltung}
\end{figure}

\subsection{Discharge}
PTC berechnung

\section{TSAL}

\subsection{Logik auf Discharge}

\subsection{Logik auf AMS Master}